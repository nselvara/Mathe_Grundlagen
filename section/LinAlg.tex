\section{Lineare Algebra}

\subsection{Wichtige Matrizen}
\textbf{Einheitsmatrix}\\
$
\left(\begin{array}{ccc}
	1 & 0 & 0\\
	0 & 1 & 0\\
	0 & 0 & 1
\end{array}\right)
$

\subsection{Matrizenprodukt}
Die \textbf{Bedingung} für ein Matrixmultiplikation ist, das die Spaltenzahl der ersten Matrix mit der Zeilenzahl der zweiten Matrix übereinstimmen muss.\\[10pt]
\fbox{Zeile mal Spalte}\\[10pt]
%
\textbf{Rechenregeln:}
\begin{compactitem}
	\item Die Faktoren dürfen nicht Vertauscht werden denn \\
	$A\cdot B \ne B\cdot A$
	\item $(A + B) \cdot C = A \cdot C + B \cdot C$
	\item $A \cdot(B \cdot C)=(A \cdot B) \cdot C$
	\item $(A \cdot B)^{T}=B^{T} \cdot A^{T} \qquad$ (Transponieren)
	\item $(A \cdot B)^{H}=B^{H} \cdot A^{H}$
	\item $\operatorname{det}(A \cdot B)=\operatorname{det}(A) \cdot \operatorname{det}(B)=\operatorname{det}(B) \cdot \operatorname{det}(A)=\operatorname{det}(B \cdot A)$
\end{compactitem}

\subsection{Skalarprodukt}
$\overrightarrow{v_{1}} \cdot \overrightarrow{v_{2}}=\left(\begin{array}{l}
x_{1} \\
y_{1} \\
z_{1}
\end{array}\right) \cdot\left(\begin{array}{l}
x_{2} \\
y_{2} \\
z_{2}
\end{array}\right)=x_{1} \cdot x_{2}+y_{1} \cdot y_{2}+z_{1} \cdot z_{2}$\\[10pt]

Der \textbf{Winkel} zwischen den Vektoren lässt sich mit folgender Formel bestimmen:\\
\fbox{$\vec{a} \cdot \vec{v}= |\vec{a}| \cdot |\vec{v}| \cdot \cos( \alpha$}\\[5pt]
Sind die Vektoren \textbf{Orthogonal} zueinander dann ist das Skalarprodukt gleich $\mathbf{0}$.


% TODO Vektorprodukt

\subsection{Gauss}
\begin{compactenum}
	\item Variable Isolieren $\circledr{*}$ (Zeile dividieren dass eine 1 heraus kommt)
	\item Variable Eliminieren $\circledb{*}$ (ein Vielfaches der Zeile von den Anderen Abziehen damit eine 0 heraus kommt.)
	\item weitermachen bis Einheitsmatrix heraus kommt.
\end{compactenum}

\textbf{Beispiel}
\begin{equation*}
	\begin{array}{rcrcrcr}
		 x &+& 2y &+& 3z &=& 10\\
		6x &+& 5y &+& 4z &=& 32\\
		 x &-&  y &+&  z &=& 2
	\end{array}
\end{equation*}
\begin{equation*}
\small
\renewcommand{\arraystretch}{1.25}
	\begin{array}{|ccc|c|}
		\hline
		\circledr{1}  &  2 & 3  & 10\\
		\circledb{6}  &  5 & 4  & 32\\
		\circledb{1}  & -1 & 1  & 2 \\
		\hline
	\end{array}
	\rightarrow
	\begin{array}{|ccc|c|}
		\hline
		1  & 2             & 3   & 10 \\
		0  & \circledr{-7} & -14 & -28\\
		0  & \circledb{-3} & -2  & -8 \\
		\hline
	\end{array}
	\rightarrow
	\begin{array}{|ccc|c|}
		\hline
		1  & 2  & \circledb{3} & 10\\
		0  & 1  & \circledb{2} & 4 \\
		0  & 0  & \circledr{4} & 4 \\
		\hline
	\end{array}
	\rightarrow
	\begin{array}{|ccc|c|}
		\hline
		1 & \circledb{2} & 0 & 7\\
		0 & 1            & 0 & 2 \\
		0 & 0            & 1 & 1 \\
		\hline
	\end{array}
	\rightarrow
	\begin{array}{|ccc|c|}
		\hline
		1 & 0 & 0 & 3\\
		0 & 1 & 0 & 2 \\
		0 & 0 & 1 & 1 \\
		\hline
	\end{array}
\end{equation*}
Die Lösung ist somit $x = 3 $,$ y = 2 $ und $ z = 1$.
\subsection{Determinante}

\subsubsection{Eigenschaften}

\begin{compactitem}
	\item Hat die Matrix $A$ eine Nullzeile/Nullspalte, dann ist die $\operatorname{det}(A) = 0$
	\item Hat $A$ zwei gleiche Zeilen/Spalten, dann ist $\operatorname{det}(A) = 0$
	\item Ist $A$ regulär, $det(A) \neq 0$\\
	Ist $A$ singulär, $det(A) = 0$
	\item Vertauscht man zwei Zeilen/Spalten, dann ändert sich das Vorzeichen der Determinante.
	\item Beschreibt eine Fläche eines Parallelogrammes (2D) bzw. ein Volumen eines Parallelepipeds (3D).
	\item Kann nur ermittelt werden, wenn die Matrix exkl. Lösungen quadratisch ist.
\end{compactitem}

\subsubsection{Wichtige Determinanten}
\begin{multicols}{2}
	\begin{equation*}
		\operatorname{det}\left(\begin{array}{ll}
		a & b \\
		c & d
		\end{array}\right)=a d-b c
	\end{equation*}
	
	\columnbreak
	
	\begin{equation*}
		\operatorname{det}\left(\begin{array}{ccc}
		a & b & c \\
		d & e & f \\
		g & h & i
		\end{array}\right)=\underbrace{a e i+b f g+c d h-c e g-a f h-b d i}_{\text {Sarrus'sche Formel }}
	\end{equation*}
\end{multicols}

\subsubsection{Determinante mit Gauss-Verfahren}
Die Determinante wird berechnet indem einfach die Pivot-Elemente multipliziert werden:
\begin{equation*}
	\left|\begin{array}{ccc}
	\circledr{2} & 2 & 4 \\
	4 & 5 & 6 \\
	7 & 8 & 5 \\
	\end{array}\right| 
	\rightarrow %^{Gauss} \rightarrow \; $$2{\;}* $$ {\;}
	2*
	\left|\begin{array}{ccc}	
	1 & 1 & 2 \\
	\circledb{4} & 5 & 6 \\
	\circledb{7} & 8 & 5 \\
	\end{array}\right|
	\rightarrow
	2*
	\left|\begin{array}{ccc}	
	1 & 1 & 2 \\
	0 & \circledr{1} & -2 \\
	0 & \circledb{1} & -9 \\
	\end{array}\right|
	\rightarrow
	2*1*
	\left|\begin{array}{ccc}	
	\circledr{1} & 1 & 2 \\
	0 & \circledr{1} & -2 \\
	0 & 0 & \circledr{-7} \\
	\end{array}\right|
	\rightarrow
	2*1*(-7)*1*1=-14
\end{equation*}
Als erstes muss die erste Zeile durch die eingekreiste Zahl dividiert werden.
Danach werden die 1. Zeile soviel mal von den Anderen abgezogen, dass überall eine 0 steht.

\subsection{Inverse}
Eine Matrix kann nur Invertiert werden wenn $\operatorname{det}(A) \ne 0$.
\subsubsection{Eigenschaften}
\begin{itemize}
	\item $A\cdot A^{-1} = E \qquad$ $E $ ist dabei die Einheitsmatrix
	\item $(A \cdot B)^{-1}=B^{-1} \cdot A^{-1}$
	\item $\left(A^{k}\right)^{-1}=\left(A^{-1}\right)^{k}$
	\item $(c A)^{-1}=c^{-1} A^{-1} \qquad c \in \mathbb{R}$
	\item $\operatorname{det}\left(A^{-1}\right)=(\operatorname{det} A)^{-1}$
\end{itemize}
\subsubsection{Formeln}
\begin{tabular}{lcl}
	\textbf{2x2 Matrix} & $\qquad$ &
		$A=\left(\begin{array}{cc}
		a & b \\
		c & d
		\end{array}\right) \rightarrow A^{-1}=\dfrac{1}{\operatorname{det}(A)}\left(\begin{array}{cc}
		d & -b \\
		-c & a
		\end{array}\right)=\dfrac{1}{a d-b c}\left(\begin{array}{cc}
		d & -b \\
		-c & a
		\end{array}\right)$\\
		
	$\quad$ & &\\
		
	\textbf{3x3 Matrix} & &
		$\left(\begin{array}{lll}
		a & b & c \\
		d & e & f \\
		g & h & i
		\end{array}\right)^{-1}=\frac{1}{\operatorname{det} A} \cdot\left(\begin{array}{ccc}
		e i-f h & c h-b i & b f-c e \\
		f g-d i & a i-c g & c d-a f \\
		d h-e g & b g-a h & a e-b d
		\end{array}\right)$\\
	
	$\quad$ & &\\
	
	\textbf{Dreiecksmatrix} & &
		$\left(\begin{array}{cccc}
		d_{11} & 0 & \cdots & 0 \\
		0 & d_{22} & \ddots & \vdots \\
		\vdots & \ddots & \ddots & 0 \\
		0 & \cdots & 0 & d_{n n}
		\end{array}\right)^{-1} = 
		\left(\begin{array}{cccc}
		d_{11}^{-1} & 0 & \cdots & 0 \\
		0 & d_{22^{-1}} & \ddots & \vdots \\
		\vdots & \ddots & \ddots & 0 \\
		0 & \cdots & 0 & d_{n n}^{-1}
		\end{array}\right)
		$
\end{tabular}

\subsection{Eigenwerte und Eigenvektoren}
Wenn eine Abbildung auf denselben Punkt fällt ($\vec{v}={\vec{v}}'$), nennt man dies Eigenfixpunkt. Der Eigenvektor $\vec{v}$ zeigt nun in
diese Richtung (als Gerade) und der Eigenwert $\lambda$ gibt den Faktor an, mit der in diese Richtung gezeigt wird.

\subsubsection{Eigenwerte berechnen}
Um die Eigenwerte zu berechnen muss die folgende Gleichung gelöst werden: $det(A-\lambda E) = 0$.\\
Die Lösung der Gleichung ($\lambda$) sind die Eigenwerte.

\textbf{Beispiel:}\\
\begin{tabular}{lcl}
	$A=\left(\begin{array}{ll}
	3 & 2 \\
	1 & 2
	\end{array}\right)$ & \quad &
	$det(A-\lambda E) = 
	\left|\begin{array}{cc}
	3-\lambda & 2 \\
	1 & 2-\lambda
	\end{array}\right| = 
	(3-\lambda)(2-\lambda)-1 \cdot 2=0$\\
	&& $=\lambda^{2}-5 \lambda+4=0 \rightarrow(\lambda-1)(\lambda-4)=0 \rightarrow \lambda_{1}=1 \rightarrow \lambda_{2}=4$
	
\end{tabular}

\subsubsection{Eigenvektoren berechnen}
Für jeden Eigenwert $\lambda_{i}$ Gleichungssystem aufstellen ($(A-\lambda_{i} E) \vec{v}_{i}=0$) und mit Gauss auflösen $\Rightarrow$ eine Zeile verschwindet $\Rightarrow \infty$ Lösungen $\Rightarrow$ Wert von verschwundener Zeile frei wählbar.\\
\textbf{Beispiel:}\\
$A=\left(\begin{array}{ll}
3 & 2 \\
1 & 2
\end{array}\right)$
\qquad
$\lambda_{1} = 1 \quad \lambda_{2} = 4$\\[5pt]

$\left( \begin{array}{cc}
	3-1 & 2\\
	1 & 2-1
\end{array} \right) \vec{v_1} = \vec{0}
\quad \rightarrow \quad
\left| \begin{array}{ccccc}
2 \cdot x_1 & + & 2 \cdot y_1 & = & 0 \\
1 \cdot x_1 & + & 1 \cdot y_1 & = & 0
\end{array} \right|
\quad \rightarrow \quad
\vec{v_1} = \left(\begin{array}{c}
	-1\\
	1
\end{array}\right)$\\[5pt]

$\left( \begin{array}{cc}
3-4 & 2\\
1 & 2-4
\end{array} \right) \vec{v_2} = \vec{0}
\quad \rightarrow \quad
\left| \begin{array}{ccccc}
-1 \cdot x_2 & + & 2 \cdot y_2 & = & 0 \\
1 \cdot x_2 & + & -2 \cdot y_2 & = & 0
\end{array} \right|
\quad \rightarrow \quad
\vec{v_2} = \left(\begin{array}{c}
2\\
1
\end{array}\right)$\\





