Gegenüber j$\omega$ bei der Fourier-Transformation ist bei der Laplace-Transformation $s$ verallgemeinert zu $s=\sigma + j\omega$. Das bedeutet, dass die Fourier-Transformierte $F(j\omega)$ durch die Laplace-Transformation $F(s)$ ausgedrückt werden kann.
\begin{minipage}{0.35\textwidth}
	\begin{framed}
		\centering
		$f(t)\laplace F(s) = \int_{0}^{\infty}f(t)e^{-st}dt$
	\end{framed}
\end{minipage}
\begin{minipage}{0.13\textwidth}
	\fbox{\centering$s=\sigma+j\omega$}
\end{minipage}
\begin{minipage}{0.52\textwidth}
	$\left\lbrace
		\begin{array}{l}
			\sigma=0 \rightarrow$Amplitude bleibt gleich$\\
			\sigma> 0 \rightarrow$Amplitude explodiert für $0<t\rightarrow\infty\\
			\sigma< 0 \rightarrow$Amplitude klingt für $0<t\rightarrow\infty$ auf $0$ ab$
		\end{array}
	\right.$
\end{minipage}
\begin{itemize}
	\item Definitionsbereich nur für kausale Systeme $t\ge0$
	\item Wachstum kleiner als der von einer Exponentialfunktion
\end{itemize}
\subsubsection{Eigenschaften der Laplacetransformation}
\renewcommand{\arraystretch}{1.5}
\begin{tabular}{|l|l|}
	\hline Linearität	& $\alpha\cdot f(t) + \beta\cdot g(t) \laplace \alpha\cdot F(s) + \beta\cdot G(s)$\\
	\hline Ähnlichkeit / Streckung im Zeitbereich & $f(\alpha t) \; \laplace \; \frac{1}{\alpha}F(\frac{s}{\alpha})\quad (\alpha \in \mathbb{R})$\\
	\hline Faltung im Zeitbereich & $f(t) * g(t) \; \laplace \; G(s) \cdot F(s)$\\
	\hline 1te Ableitung im Zeitbereich & $\frac{\partial}{\partial t}f(t) \; \laplace \; sF(s) - f(0^{+})$\\
	\hline 2te Ableitung im Zeitbereich & $\frac{\partial^{2}}{\partial t^{2}}f(t) \; \laplace \; s^{2}F(s) - sf(0^{+}) - f'(0^{+})$\\
	\hline nte Ableitung im Zeitbereich & $\frac{\partial^{n}f(t)}{\partial t^{n}} \; \laplace \; s^{n}F(s) - s^{n-1}f(0^{+}) - s^{n-1}\frac{\partial f(0^{+})}{\partial t}- \ldots - s^{0}\frac{\partial^{n-1}f(0^{+})}{\partial t^{n-1}}$\\
	\hline Multiplikation mit t & $t \cdot f(t) \; \laplace \; \frac{-\partial F(s)}{\partial s}$\\
	\hline Ableitung im Frequenzbereich & $(-t)^{n} f(t) \; \laplace \; \frac{\partial^{n} F(s)}{(\partial s)^{n}}$\\
	\hline Verschiebung im Zeitbereich nach rechts & $\sigma(t-a)f(t - a) \; \laplace \; F(s)*e^{-as}$ \\
	\hline Verschiebung im Zeitbereich nach links & $\sigma(t-a)f(t + a) \; \laplace \; e^{as} \cdot [F(s) - \int\limits_0^{a} f(t) \cdot e^{-st} dt]$\\
	\hline Verschiebung im Frequenzbereich (Dämpfungssatz) & $f(t)e^{\pm\alpha t} \; \laplace \; F(s\mp\alpha)$ \\
	\hline Integration im Originalbereich (Sprungantwort)& $\int\limits_0^t f(u)du \; \laplace \; \frac{1}{s}\cdot F(s)$ \\
	\hline Anfangswert & $\lim_{t\rightarrow 0^+} f(t) = \lim_{s\rightarrow \infty} sF(s),\text{~wenn
	}  \lim_{t\rightarrow 0} f(t)\text{~existiert}.$ \\
	\hline Endwert & $\lim_{t\rightarrow \infty} f(t) = \lim_{s\rightarrow 0} sF(s),\text{~wenn
	}  \lim_{t\rightarrow \infty} f(t)\text{~existiert}.$ \\
	\hline
\end{tabular}
\renewcommand{\arraystretch}{1}
\subsubsection{Laplace-Tabelle}
\renewcommand{\arraystretch}{2}
\begin{minipage}{0.465\textwidth}
	\begin{center}
		\begin{tabular}{p{4cm}p{0.75cm}p{3cm}}
			$\sigma \left( t \right)$ & $\; \laplace \;$ & $\dfrac{1}{s}$ \\
			
			$\sigma \left( t \right) \cdot t$ & $\; \laplace \;$ & $\dfrac{1}{s^2}$\\
			
			$\sigma \left( t \right) \cdot t^2$ & $\; \laplace \;$ & $\dfrac{2}{s^3}$\\
			
			$\sigma \left( t \right) \cdot t^n$ & $\; \laplace \;$ & $\dfrac{n!}{s^{n+1}}$\\
			
			$\sigma \left( t \right) \cdot e^{\alpha t}$ & $\; \laplace \;$ &
			$\dfrac{1}{s-\alpha}$\\
			
			$\sigma \left( t \right) \cdot t \cdot e^{\alpha t}$ & $\; \laplace \;$ & $\dfrac{1}{( s - \alpha )^2}$\\
			
			$\sigma \left( t \right)\cdot t^2 \cdot e^{\alpha t}$ &
			$\; \laplace \;$ & $\dfrac{2}{{( s - \alpha )}^3}$\\
			
			$\sigma \left( t \right)\cdot t^n \cdot e^{ \alpha t}$ &
			$\; \laplace \;$ & $\dfrac{n!}{(s-\alpha)^{n+1}}$\\
			
			$\sigma \left( t \right)\cdot \dfrac { 1 - e ^ { - \alpha t } } { \alpha }$ & $\; \laplace \;$ & $\dfrac { 1 } { s ( s + \alpha ) }$\\
			
			$\sigma \left( t \right)\cdot \dfrac {e ^ { - \alpha t }+\alpha t -1 } { \alpha^2 }$ & $\; \laplace \;$ & $\dfrac { 1 } { s^2 ( s + \alpha ) }$\\
			
			$\sigma \left( t \right)\cdot \dfrac {1- e ^ { - \alpha t } - \alpha t e ^ {- \alpha t }}{ \alpha ^2 }$ & $\; \laplace \;$ & $\dfrac { 1 } { s ( s + \alpha )^2 }$\\		
		\end{tabular}
	\end{center}
\end{minipage}
\begin{minipage}{0.5\textwidth}
	\begin{center}
		\begin{tabular}{p{5cm}p{0.75cm}p{3cm}}
			
			$\delta \left( t \right)$ & $\; \laplace \;$ & $1\left( s \right)$ \\
			
			$\delta \left( t - \alpha \right)$ & $\; \laplace \;$ & $e^{- \alpha s}$\\
			
			$\sigma\left( t - \alpha \right)$ & $\; \laplace \;$ & $ \dfrac{1}{s} \cdot e^{- \alpha s}$\\
			
			$\sigma \left( t \right) \cdot \sin \left(\omega t \right)$ & $\; \laplace \;$ &
			$\dfrac{\omega}{s^2 + {\omega^2}}$\\
			
			$\sigma \left( t \right) \cdot \cos \left( \omega t \right)$ & $\; \laplace \;$ &
			$\dfrac{s}{ s^2 + \omega^2}$\\
			
			$\sigma \left( t \right) \cdot  e^{ \alpha t} \cdot \sin \left(\omega t \right)$ & $\; \laplace \;$ 
			& 	$\dfrac{\omega}{(s-a)^2 + {\omega^2}}$\\
			$\sigma \left( t \right) \cdot e^{ \alpha t} \cdot \cos \left( \omega t \right) $ & $\; \laplace \;$ &
			$\dfrac{s-a}{(s-a)^2 + \omega^2}$\\
			
			$\sigma \left( t \right)\cdot t \cdot \dfrac{\sin \left( \alpha t \right)} { 2 \alpha }$ & $\; \laplace \;$ & $\dfrac{s}{ \left(s^ {2}+ \alpha ^{2} \right)^{2}}$ \\
			
			$\sigma \left( t \right)\cdot \dfrac { e ^ { - \alpha t } - e ^ { - \beta t } } { \beta - \alpha }$ & $\; \laplace \;$ & $\dfrac { 1 } { ( s + \alpha ) ( s + \beta ) }$\\
			
			$\sigma \left( t \right)\cdot \dfrac {(\alpha - \beta) +\beta e ^ { - \alpha t } - \alpha e ^ { - \beta t } } { \alpha \beta (\alpha - \beta) }$ & $\; \laplace \;$ & $\dfrac { 1 } {s ( s + \alpha ) ( s + \beta ) }$\\
			
			$\sigma \left( t \right)\cdot \dfrac { e ^ { - \beta t } ( \alpha \cos ( \alpha t ) - \beta \sin ( \alpha t ) ) } { \alpha }$& $\; \laplace \;$ & $\dfrac { s } { ( s + \beta ) ^ { 2 } + \alpha ^ { 2 } }$\\
		\end{tabular}
	\end{center}
\end{minipage}
\renewcommand{\arraystretch}{1}