Die $\delta$-Funktion ist auch als Impulsfunktion oder DIRAC bekannt.
\subsubsection[Definition]{Definition}
\begin{minipage}{0.3\textwidth}
		\begin{framed}
			\centering
			$\delta(t)=0$ für alle $t \neq 0$\\
			$\int_{-\infty}^{\infty} \delta(t) d t=1$\\
			$\int_{-\infty}^{\infty} \delta(t) \cdot \varphi(t) d t=\varphi(0)$
		\end{framed}
\end{minipage}\\[10pt]
%
\begin{minipage}{0.5\textwidth}
	\subsubsection{Verschiebung}
	$\int_{-\infty}^{\infty} \delta\left(t-t_{0}\right) \varphi(t) d t=\varphi\left(t_{0}\right)$
\end{minipage}
%
\begin{minipage}{0.5\textwidth}
	\subsubsection{Multiplikation}
	$s(t) \delta\left(t-t_{0}\right)=s\left(t_{0}\right) \delta\left(t-t_{0}\right)$
\end{minipage}\\[10pt]
%
\begin{minipage}{0.5\textwidth}
	\subsubsection{Ableitung der Deltafunktion}
	$\int_{-\infty}^{\infty} \delta^{(k)}\left(t-t_{0}\right) f(t) d t=(-1)^{k} f^{(k)}\left(t_{0}\right)$
\end{minipage}
\begin{minipage}{0.5\textwidth}
	\subsubsection{Ableitung des Einheitssprung}
	$\dot{\sigma}(t)=\delta(t)$
\end{minipage}
