\begin{tabular}{|l|l|l|m{8.8cm}|}
	\hline
	\textbf{Majuskel}	& \textbf{Minuskel}			& \textbf{Aussprache}		& \textbf{Häufigste Verwendung in Technik} \\[3pt]
	\hline
	$A$ 				& $\alpha$					& Alpha						& $\alpha =$ Winkel, Winkelbeschleunigung, allgemeiner Faktor \\[3pt]
	$B$ 				& $\beta$					& (Alt: Beta), (Neu: Feta)	& $\beta =$ Winkel, Bandbreite\\[3pt]
	$\Gamma$			& $\gamma$					& Gamma						& $\Gamma =$ Reflexionsfaktor, $\gamma =$ Winkel \\[3pt]
	$\Delta$			& $\delta$					& Delta						& $\Delta =$ Differenz, $\delta =$ Dirac-Funktion \\[3pt]
	$E$					& $\epsilon$, $\varepsilon$	& Epsilon 					& $\varepsilon =$ Permittivität \\[3pt]
	$Z$					& $\zeta$					& Zeta 						& $\zeta =$ Dämpfungsgrad \\[3pt]
	$H$					& $\eta$					& Eta 						& $\eta =$ Wirkungsgrad, Rauschleistung \\[3pt]
	$\Theta$			& $\theta$, $\vartheta$		& Theta 					& $\Theta =$ Magnetische Durchflutung, $\theta =$ Winkel, Wärmewiderstand, $\vartheta =$ (Celsius-)Temperatur \\[3pt]
	$I$					& $\iota$					& Iota 						&  \\[3pt]
	$K$					& $\kappa$, $\varkappa$		& Kappa 					& $\kappa =$ Gaskonstante \\[3pt]
	$\Lambda$			& $\lambda$					& Lambda 					& $\Lambda =$ Logarithmisches Dekrement, $\lambda =$ Wellenlänge, Eigenwerte \\[3pt]
	$M$					& $\mu$						& Mu 						& $\mu =$ Permeabilität \\[3pt]
	$M$					& $\nu$						& Nu 						& $\nu =$ Frequenz, Poisson-Zahl \\[3pt]
	$\Xi$				& $\xi$						& Xi 						& $\xi =$ Störglied in der Analysis, Zufallsvariable \\[3pt]
	$O$ 				& $o$ 						& Omikron 					&  \\[3pt]
	$\Pi$				& $\pi$, $\varpi$			& Pi, Bi					& $\Pi =$ Produktzeichen, $\pi =$ Kreiszahl \\[3pt]
	$P$					& $\rho$, $\varrho$			& Rho 						& $\rho =$ Spezifischer Widerstand, (Ladungs-)Dichte \\[3pt]
	$\Sigma$			& $\sigma$, $\varsigma$		& Sigma 					& $\Sigma =$ Summenzeichen, $\sigma =$ Elektrische Leitfähigkeit, Standardabweichung \\[3pt]
	$T$					& $\tau$					& Tau 						& $\tau =$ Zeitkonstante der passiven Reaktanz-Elemente \\[3pt]
	$\Upsilon$			& $\upsilon$				& Ypsilon 					&  \\[3pt]
	$\Phi$				& $\phi$, $\varphi$			& Phi 						& $\Phi =$ Magnetischer Fluss, $\phi =$ Winkel, $\varphi =$ (Polarkoordinaten) Winkel, Phasenverschiebung, elektrisches Potential \\[3pt]
	$X$					& $\chi$					& Chi 						&  \\[3pt]
	$\Psi$				& $\psi$					& Psi 						& $\Psi =$ Elektrischer Fluss \\[3pt]
	$\Omega$			& $\omega$					& Omega 					& $\Omega =$ Widerstand-, Impedanz-Einheit, $\omega =$ Kreisfrequenz \\[3pt]
	\hline
\end{tabular}